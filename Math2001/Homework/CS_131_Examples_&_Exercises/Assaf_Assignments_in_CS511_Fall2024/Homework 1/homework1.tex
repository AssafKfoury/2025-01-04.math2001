\documentclass[12pt,leqno,fleqn]{article}

\usepackage{tikz}
\usepackage{tikz-qtree,tikz-qtree-compat} % for parse trees

\usepackage{latexsym}
\usepackage{amssymb}
\usepackage{amsmath}
\usepackage{amsthm}
\usepackage{amsfonts}

% \usepackage{prooftree}
\usepackage{flagderiv}
\usepackage{logicproof}
\usepackage{bussproofs}
% WARNING:
% Do NOT use package `bussproofs' and package `prooftree' at the same time,
% \begin{prooftree} ... \end{prooftree} is an environment defined in the
% package  `bussproofs', which conflicts with the name of the package
% `prooftree'.

\newcommand{\N}{\mathbb{N}}
\newcommand{\Z}{\mathbb{Z}}

\newcommand{\defn}{\overset{\text{\scriptsize def}}{=}}
\newcommand{\Intro}[1]{{#1}{\text{I}}}
\newcommand{\IntroA}[1]{{#1}{\text{I}_1}}
\newcommand{\IntroB}[1]{{#1}{\text{I}_2}}
\newcommand{\Elim}[1]{{#1}{\text{E}}}
\newcommand{\ElimA}[1]{{#1}{\text{E}_1}}
\newcommand{\ElimB}[1]{{#1}{\text{E}_2}}
\newcommand{\Set}[1]{\{ #1 \}}
\newcommand{\SET}[1]{\Bigl\{ #1 \Bigr\}}

\newenvironment{theorem}[2][Theorem]{\begin{trivlist}
\item[\hskip \labelsep {\bfseries #1}\hskip \labelsep {\bfseries #2.}]}{\end{trivlist}}
\newenvironment{lemma}[2][Lemma]{\begin{trivlist}
\item[\hskip \labelsep {\bfseries #1}\hskip \labelsep {\bfseries #2.}]}{\end{trivlist}}
\newenvironment{exercise}[2][Exercise]{\begin{trivlist}
\item[\hskip \labelsep {\bfseries #1}\hskip \labelsep {\bfseries #2.}]}{\end{trivlist}}
\newenvironment{problem}[2][Problem]{\begin{trivlist}
\item[\hskip \labelsep {\bfseries #1}\hskip \labelsep {\bfseries #2.}]}{\end{trivlist}}
\newenvironment{question}[2][Question]{\begin{trivlist}
\item[\hskip \labelsep {\bfseries #1}\hskip \labelsep {\bfseries #2.}]}{\end{trivlist}}
\newenvironment{corollary}[2][Corollary]{\begin{trivlist}
\item[\hskip \labelsep {\bfseries #1}\hskip \labelsep {\bfseries #2.}]}{\end{trivlist}}

\newenvironment{solution}{\begin{proof}[Solution]}{\end{proof}}
 
\begin{document}
 
\title{Homework 1 Solutions}
\author{CS 511 Formal Methods}

\maketitle

\section*{Exercise 1.2.1}

\subsection*{(h) $p \vdash (p \to q) \to q$}

\medskip 
\begin{logicproof}{3}
    p                   & premise \\
    \begin{subproof}
        p \to q         & assumption \\
        q               & $\Elim{\to}$ 1,2 
    \end{subproof}
    (p \to q) \to q     & $\Intro{\to}$ 2--3 
\end{logicproof}

\subsection*{(i) $(p \to r) \land (q \to r) \vdash p \land q \to r$}

\medskip
\begin{logicproof}{3}
    (p \to r) \wedge (q \to r)              & premise \\
    p \to r                                 & $\ElimA{\land}$ 1 \\
    \begin{subproof}
        p \wedge q                          & assumption \\
        p                                   & $\ElimA{\land}$ 3 \\
        r                                   & $\Elim{\to}$ 2,4
    \end{subproof}
    p \wedge q \to r                        & $\Intro{\to}$ 3--5 
\end{logicproof}

\subsection*{(j) $q \to r \vdash (p \to q) \to (p \to r)$}

\begin{logicproof}{3}
    q \to r                                 & premise \\
    \begin{subproof}
        p \to q                             & assumption \\
        \begin{subproof}
            p                               & assumption \\ 
            q                               & $\Elim{\to}$ 2,3 \\
            r                               & $\Elim{\to}$ 1,4 
        \end{subproof}
        p \to r                             & $\Intro{\to}$ 3--5 
    \end{subproof}
    (p \to q) \to (p \to r)                 & $\Intro{\to}$ 2--6 
\end{logicproof}

\section*{Exercise 1.4.2}

\subsection*{(g)}

\begin{tabular}{ll|l|l|l}
    $p$ & $q$ & $p \to q$ & $(p \to q) \to p$ & $((p \to q) \to p) \to p$ \\ \hline
    T   & T   & T         & T                 & T                         \\
    T   & F   & F         & T                 & T                         \\
    F   & T   & T         & F                 & T                         \\
    F   & F   & T         & F                 & T                        
\end{tabular}

\iffalse 
\begin{tabular}{@{ }c@{ }@{ }c | c@{ }@{}c@{}@{}c@{}@{ }c@{ }@{ }c@{ }@{ }c@{ }@{}c@{}@{ }c@{ }@{ }c@{ }@{}c@{}@{ }c@{ }@{ }c@{ }@{ }c}
    p & q &  & ( & ( & p & $\rightarrow$ & q & ) & $\rightarrow$ & p & ) & $\rightarrow$ & p & \\
    \hline 
    T & T &  &  &  & T & T & T &  & T & T &  & \textcolor{red}{T} & T & \\
    T & F &  &  &  & T & F & F &  & T & T &  & \textcolor{red}{T} & T & \\
    F & T &  &  &  & F & T & T &  & F & F &  & \textcolor{red}{T} & F & \\
    F & F &  &  &  & F & T & F &  & F & F &  & \textcolor{red}{T} & F & \\
    \end{tabular}
\fi 

\subsection*{(h)}

\scriptsize
\begin{tabular}{lll|l|l|l|l|l|l}
    $p$ & $q$ & $r$ & $p \lor q$ & $p \to r$ & $q \to r$ & $(p \lor q) \to r$ & $(p \to r) \lor (q \to r)$ & $((p \lor  q) \to r) \to ((p \to r) \lor (q \to r))$ \\ \hline
    T   & T   & T   & T          & T         & T         & T                  & T                          & T                                                    \\
    T   & T   & F   & T          & F         & F         & F                  & F                          & T                                                    \\
    T   & F   & T   & T          & T         & T         & T                  & T                          & T                                                    \\
    T   & F   & F   & T          & F         & T         & F                  & T                          & T                                                    \\
    F   & T   & T   & T          & T         & T         & T                  & T                          & T                                                    \\
    F   & T   & F   & T          & T         & F         & F                  & T                          & T                                                    \\
    F   & F   & T   & F          & T         & T         & T                  & T                          & T                                                    \\
    F   & F   & F   & F          & T         & T         & T                  & T                          & T                                                   
\end{tabular}
\normalsize

\iffalse 
\begin{tabular}{@{ }c@{ }@{ }c@{ }@{ }c | c@{ }@{}c@{}@{}c@{}@{ }c@{ }@{ }c@{ }@{ }c@{ }@{}c@{}@{ }c@{ }@{ }c@{ }@{}c@{}@{ }c@{ }@{}c@{}@{}c@{}@{ }c@{ }@{ }c@{ }@{ }c@{ }@{}c@{}@{ }c@{ }@{}c@{}@{ }c@{ }@{ }c@{ }@{ }c@{ }@{}c@{}@{}c@{}@{ }c}
    p & q & r &  & ( & ( & p & $\lor$ & q & ) & $\rightarrow$ & r & ) & $\rightarrow$ & ( & ( & p & $\rightarrow$ & r & ) & $\lor$ & ( & q & $\rightarrow$ & r & ) & ) & \\
    \hline 
    T & T & T &  &  &  & T & T & T &  & T & T &  & \textcolor{red}{T} &  &  & T & T & T &  & T &  & T & T & T &  &  & \\
    T & T & F &  &  &  & T & T & T &  & F & F &  & \textcolor{red}{T} &  &  & T & F & F &  & F &  & T & F & F &  &  & \\
    T & F & T &  &  &  & T & T & F &  & T & T &  & \textcolor{red}{T} &  &  & T & T & T &  & T &  & F & T & T &  &  & \\
    T & F & F &  &  &  & T & T & F &  & F & F &  & \textcolor{red}{T} &  &  & T & F & F &  & T &  & F & T & F &  &  & \\
    F & T & T &  &  &  & F & T & T &  & T & T &  & \textcolor{red}{T} &  &  & F & T & T &  & T &  & T & T & T &  &  & \\
    F & T & F &  &  &  & F & T & T &  & F & F &  & \textcolor{red}{T} &  &  & F & T & F &  & T &  & T & F & F &  &  & \\
    F & F & T &  &  &  & F & F & F &  & T & T &  & \textcolor{red}{T} &  &  & F & T & T &  & T &  & F & T & T &  &  & \\
    F & F & F &  &  &  & F & F & F &  & T & F &  & \textcolor{red}{T} &  &  & F & T & F &  & T &  & F & T & F &  &  & \\
    \end{tabular}
\fi 

\subsection*{(i)}

\begin{tabular}{ll|l|l|l|l|l}
    $p$ & $q$ & $\neg p$ & $\neg q$ & $p \to q$ & $\neg p \to \neg q$ & $(p \to q) \to (\neg p \to \neg q)$ \\ \hline
    T   & T   & F        & F        & T         & T                   & T                                   \\
    T   & F   & F        & T        & F         & T                   & T                                   \\
    F   & T   & T        & F        & T         & F                   & F                                   \\
    F   & F   & T        & T        & T         & T                   & T                                  
\end{tabular}

\iffalse 
\begin{tabular}{@{ }c@{ }@{ }c | c@{ }@{}c@{}@{ }c@{ }@{ }c@{ }@{ }c@{ }@{}c@{}@{ }c@{ }@{}c@{}@{ }c@{ }@{ }c@{ }@{ }c@{ }@{ }c@{ }@{ }c@{ }@{}c@{}@{ }c}
    p & q &  & ( & p & $\rightarrow$ & q & ) & $\rightarrow$ & ( & $\neg$ & p & $\rightarrow$ & $\neg$ & q & ) & \\
    \hline 
    T & T &  &  & T & T & T &  & \textcolor{red}{T} &  & F & T & T & F & T &  & \\
    T & F &  &  & T & F & F &  & \textcolor{red}{T} &  & F & T & T & T & F &  & \\
    F & T &  &  & F & T & T &  & \textcolor{red}{F} &  & T & F & F & F & T &  & \\
    F & F &  &  & F & T & F &  & \textcolor{red}{T} &  & T & F & T & T & F &  & \\
    \end{tabular}
\fi 

\section*{Exercise 1.5.3}

\subsection*{(b)}

Show that, if $C \subseteq \{\neg, \land, \lor, \to, \bot\}$ is adequate for propositional logic, then $\neg \in C$ or $\bot \in C$.  (Hint:  suppose $C$ contains neither $\neg$ nor $\bot$ and consider the truth value of a formula $\phi$, formed by using only the connectives in $C$, for a valuation in which every atom is assigned $\top$.)

\begin{proof}
    We prove the contrapositive of the original statement.  Assume that neither $\neg$ nor $\bot$ are in $C$.  I claim that $C \subseteq \{\land, \lor, \to\}$ is not adequate for propositional logic because it can make neither $\neg$ nor $\bot$ from $\land$, $\lor$, and $\to$.  To see this, observe the following truth table:  

    \begin{tabular}{ll|l|l|l|l|l|l}
        $p$ & $q$ & $p \to q$ & $q \to p$ & $p \lor q$ & $p \land q$ & $\neg p$ & $\bot$ \\ \hline 
        T   & T   & T         & T         & T          & T           & F        & F \\
        T   & F   & F         & T         & T          & F           & F        & F \\
        F   & T   & T         & F         & T          & F           & T        & F \\
        F   & F   & T         & T         & F          & F           & T        & F 
    \end{tabular}

    \iffalse 
    \begin{tabular}{@{ }c@{ }@{ }c | c@{ }@{}c@{}@{}c@{}@{}c@{}@{ }c@{ }@{ }c@{ }@{ }c@{ }@{}c@{}@{ }c@{ }@{}c@{}@{ }c@{ }@{ }c@{ }@{ }c@{ }@{}c@{}@{}c@{}@{ }c@{ }@{}c@{}@{ }c@{ }@{ }c@{ }@{ }c@{ }@{}c@{}@{}c@{}@{ }c@{ }@{}c@{}@{ }c@{ }@{ }c@{ }@{ }c@{ }@{}c@{}@{ }c}
        p & q &  & ( & ( & ( & p & $\rightarrow$ & q & ) & $\lor$ & ( & q & $\rightarrow$ & p & ) & ) & $\lor$ & ( & p & $\lor$ & q & ) & ) & $\lor$ & ( & p & $\land$ & q & ) & \\
        \hline 
        T & T &  &  &  &  & T & \textcolor{red}{T} & T &  &  &  & T & \textcolor{red}{T} & T &  &  &  &  & T & \textcolor{red}{T} & T &  &  &  &  & T & \textcolor{red}{T} & T &  & \\
        T & F &  &  &  &  & T & \textcolor{red}{F} & F &  &  &  & F & \textcolor{red}{T} & T &  &  &  &  & T & \textcolor{red}{T} & F &  &  &  &  & T & \textcolor{red}{F} & F &  & \\
        F & T &  &  &  &  & F & \textcolor{red}{T} & T &  &  &  & T & \textcolor{red}{F} & F &  &  &  &  & F & \textcolor{red}{T} & T &  &  &  &  & F & \textcolor{red}{F} & T &  & \\
        F & F &  &  &  &  & F & \textcolor{red}{T} & F &  &  &  & F & \textcolor{red}{T} & F &  &  &  &  & F & \textcolor{red}{F} & F &  &  &  &  & F & \textcolor{red}{F} & F &  & \\
    \end{tabular}
    \fi 

    \noindent Note that, in the case where $p$ is true, none of the connectives in $C \subseteq \{\land, \lor, \to\}$ with an arbitrary Boolean variable $q$ can turn the truth value of the statement to false.  For both $\neg$ and $\bot$ in the case where $p$ is true, they change the truth value to false.  We can look at the partial truth table to emphasize this:  

    \begin{tabular}{ll|l|l|l|l|l|l}
        $p$ & $q$ & $p \to q$ & $q \to p$ & $p \lor q$ & $p \land q$ & $\neg p$ & $\bot$ \\ \hline 
        T   & T   & T         & T         & T          & T           & F        & F \\
    \end{tabular}
    
    \noindent This means that no combination of $\to$, $\lor$, and $\land$ can create $\neg$ or $\bot$, which implies that $C \subseteq \{\land, \lor, \to\}$ is not adequate for propositional logic. (This is an informal description of structural induction, but I wanted to go easy on its use here since we have not formally covered the technique in class, yet.) 
    
\end{proof}

\subsection*{(c)}

Is $\{\leftrightarrow, \neg\}$ adequate?  Prove your answer.  

\begin{proof}
    $\{\leftrightarrow, \neg\}$ is not adequate for propositional logic.  This can be seen easiest from looking at a truth table:  

    \begin{tabular}{ll|l|l}
        $p$ & $q$ & $\neg p$ & $p \to q$ \\ \hline 
        T   & T   & F        & T         \\
        T   & F   & F        & F         \\
        F   & T   & T        & T         \\
        F   & F   & T        & T        
    \end{tabular}

    \begin{tabular}{ll|l|l}
        $p$ & $q$ & $p \leftrightarrow q$ & $p \to q$ \\ \hline 
        T   & T   & T                     & T         \\
        T   & F   & F                     & F         \\
        F   & T   & F                     & T         \\
        F   & F   & T                     & T        
    \end{tabular}

    \iffalse 
    \begin{tabular}{@{ }c@{ }@{ }c | c@{ }@{}c@{}@{ }c@{ }@{ }c@{ }@{ }c@{ }@{}c@{}@{ }c@{ }@{ }c@{ }@{ }c@{ }@{}c@{}@{}c@{}@{ }c@{ }@{}c@{}@{ }c@{ }@{ }c@{ }@{ }c@{ }@{}c@{}@{ }c}
        p & q &  & ( & $\lnot$ & p & $\lor$ & ( & p & $\leftrightarrow$ & q & ) & ) & $\lor$ & ( & p & $\rightarrow$ & q & ) & \\
        \hline 
        T & T &  &  & \textcolor{red}{F} & T & T &  & T & \textcolor{red}{T} & T &  &  & T &  & T & \textcolor{red}{T} & T &  & \\
        T & F &  &  & \textcolor{red}{F} & T & F &  & T & \textcolor{red}{F} & F &  &  & F &  & T & \textcolor{red}{F} & F &  & \\
        F & T &  &  & \textcolor{red}{T} & F & T &  & F & \textcolor{red}{F} & T &  &  & T &  & F & \textcolor{red}{T} & T &  & \\
        F & F &  &  & \textcolor{red}{T} & F & T &  & F & \textcolor{red}{T} & F &  &  & T &  & F & \textcolor{red}{T} & F &  & \\
    \end{tabular}
    \fi 

    \noindent The connectives $\neg$ and $\leftrightarrow$ have an even number of true values and $\to$ has an odd number of true values.  This means that any combination of $\neg$ and $\leftrightarrow$ cannot form $\to$ as functions of variables $p$ and $q$ (do you see why?).  
    
\end{proof}

\end{document}