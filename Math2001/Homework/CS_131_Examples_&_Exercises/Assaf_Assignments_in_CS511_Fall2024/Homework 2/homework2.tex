\documentclass[12pt,leqno,fleqn]{article}

\usepackage{tikz}
\usepackage{tikz-qtree,tikz-qtree-compat} % for parse trees

\usepackage{latexsym}
\usepackage{amssymb}
\usepackage{amsmath}
\usepackage{amsthm}
\usepackage{amsfonts}

% \usepackage{prooftree}
\usepackage{flagderiv}
\usepackage{logicproof}
\usepackage{bussproofs}
% WARNING:
% Do NOT use package `bussproofs' and package `prooftree' at the same time,
% \begin{prooftree} ... \end{prooftree} is an environment defined in the
% package  `bussproofs', which conflicts with the name of the package
% `prooftree'.

\newcommand{\N}{\mathbb{N}}
\newcommand{\Z}{\mathbb{Z}}

\newcommand{\defn}{\overset{\text{\scriptsize def}}{=}}
\newcommand{\Intro}[1]{{#1}{\text{I}}}
\newcommand{\IntroA}[1]{{#1}{\text{I}_1}}
\newcommand{\IntroB}[1]{{#1}{\text{I}_2}}
\newcommand{\Elim}[1]{{#1}{\text{E}}}
\newcommand{\ElimA}[1]{{#1}{\text{E}_1}}
\newcommand{\ElimB}[1]{{#1}{\text{E}_2}}
\newcommand{\Set}[1]{\{ #1 \}}
\newcommand{\SET}[1]{\Bigl\{ #1 \Bigr\}}

\newenvironment{theorem}[2][Theorem]{\begin{trivlist}
\item[\hskip \labelsep {\bfseries #1}\hskip \labelsep {\bfseries #2.}]}{\end{trivlist}}
\newenvironment{lemma}[2][Lemma]{\begin{trivlist}
\item[\hskip \labelsep {\bfseries #1}\hskip \labelsep {\bfseries #2.}]}{\end{trivlist}}
\newenvironment{exercise}[2][Exercise]{\begin{trivlist}
\item[\hskip \labelsep {\bfseries #1}\hskip \labelsep {\bfseries #2.}]}{\end{trivlist}}
\newenvironment{problem}[2][Problem]{\begin{trivlist}
\item[\hskip \labelsep {\bfseries #1}\hskip \labelsep {\bfseries #2.}]}{\end{trivlist}}
\newenvironment{question}[2][Question]{\begin{trivlist}
\item[\hskip \labelsep {\bfseries #1}\hskip \labelsep {\bfseries #2.}]}{\end{trivlist}}
\newenvironment{corollary}[2][Corollary]{\begin{trivlist}
\item[\hskip \labelsep {\bfseries #1}\hskip \labelsep {\bfseries #2.}]}{\end{trivlist}}

\newenvironment{solution}{\begin{proof}[Solution]}{\end{proof}}

\begin{document}
 
\title{Homework 2 Solutions}
\author{CS 511 Formal Methods}

\maketitle

\begin{exercise}{1:  Lecture Slides 06, Page 9}
    For all strings $s, t \in A^*$, it holds that $$\textbf{reverse}(s \cdot t) = \textbf{reverse}(t) \cdot \textbf{reverse}(s)$$
\end{exercise}

\begin{proof}
    Suppose $s \in A^*$ is an arbitrary string.  We prove by structural induction on $t \in A^*$.  First assume that $t = \varepsilon$, that is, $t$ is the empty string.  Then 

    \begin{equation*}
        \begin{aligned}
            \textbf{reverse}(s \cdot t) &= \textbf{reverse}(s \cdot \varepsilon) \\
            &= \textbf{reverse}(s) \\
            &= \varepsilon \cdot \textbf{reverse}(s) \\
            &= \textbf{reverse}(\varepsilon) \cdot \textbf{reverse}(s) \\
            &= \textbf{reverse}(t) \cdot \textbf{reverse}(s) 
        \end{aligned}
    \end{equation*}

    Next, assume $t = t' \cdot x$ where $t', s \in A^*$, $x \in A$, and $\textbf{reverse}(s \cdot t') = \textbf{reverse}(t') \cdot \textbf{reverse}(s)$.  Then 

    \begin{equation*}
        \begin{aligned}
            \textbf{reverse}(s \cdot t) &= \textbf{reverse}(s \cdot (t' \cdot x)) \\
            &= \textbf{reverse}((s \cdot t') \cdot x) \\ 
            &= x \cdot \textbf{reverse}(s \cdot t') \\
            &= x \cdot \textbf{reverse}(t') \cdot \textbf{reverse}(s) \\
            &= \textbf{reverse}(t' \cdot x) \cdot \textbf{reverse}(s) \\ 
            &= \textbf{reverse}(t) \cdot \textbf{reverse}(s) 
        \end{aligned}
    \end{equation*}

    By the principle of structural induction, we have that $\textbf{reverse}(s \cdot t) = \textbf{reverse}(t) \cdot \textbf{reverse}(s)$ for all $s, t \in A^*$.  
\end{proof}

\begin{exercise}{2:  LCS Exercise 1.4.15}
    Use mathematical induction on $n$ to prove the theorem $$(\varphi_{1} \wedge ( \varphi_{2} \wedge \cdots (\varphi_{n-1} \wedge \varphi_{n}) \cdots )) \to \psi \vdash \varphi_{1} \to (\varphi_{2} \to \cdots (\varphi_{n-1} \to (\varphi_{n} \to \psi)))$$
\end{exercise}

\begin{proof}
    \iffalse
    The base case is $\varphi_{1} \to \psi \vdash \varphi_{1} \to \psi$ and thus is trivial.  Assume that this holds for $n-1$.  That is, we have a proof 
    
    \begin{logicproof}{3}
        (\varphi_{1} \wedge ( \varphi_{2} \wedge \cdots (\varphi_{n-2} \wedge \varphi_{n-1}) \cdots )) \to \psi \\ 
        \vdots \\ 
        \varphi_{1} \to (\varphi_{2} \to \cdots (\varphi_{n-2} \to (\varphi_{n-1} \to \psi))) 
    \end{logicproof}

    Then we have (ignore the numbering)
    \fi
    We prove directly instead.  
    \begin{logicproof}{3}
        (\varphi_{1} \wedge ( \varphi_{2} \wedge \cdots (\varphi_{n-1} \wedge \varphi_{n}) \cdots )) \to \psi & premise \\ 
        \begin{subproof}
            \varphi_{1} & assumption \\
            \begin{subproof}
                \vdots \\
                \begin{subproof}
                    \varphi_{n} & assumption \\
                    \varphi_{n-1} \wedge \varphi_{n} & $\Intro{\wedge}$ \\
                    \vdots & \\ 
                    (\varphi_{1} \wedge ( \varphi_{2} \wedge \cdots (\varphi_{n-1} \wedge \varphi_{n}) \cdots )) & $\Intro{\wedge}$ \\
                    \psi & $\Elim{\to}$ 
                \end{subproof}
                \varphi_{n} \to \psi & $\Intro{\to}$ 
            \end{subproof}
            \vdots & 
        \end{subproof}
        \varphi_{1} \to (\varphi_{2} \to \cdots (\varphi_{n-1} \to (\varphi_{n} \to \psi)))
    \end{logicproof}

    \iffalse 
    First, I'll make a small detour to prove the associativity and commutativity of $\wedge$.  

    $$(p \wedge q) \wedge r \vdash p \wedge (q \wedge r)$$

    \begin{logicproof}{3}
        (p \wedge q) \wedge r & premise \\
        p \wedge q & $\ElimA{\wedge}$ 1 \\ 
        r & $\ElimB{\wedge}$ 1 \\
        p & $\ElimA{\wedge}$ 2 \\ 
        q & $\ElimA{\wedge}$ 2 \\
        q \wedge r & $\Intro{\wedge}$ 3,5 \\
        p \wedge (q \wedge r) & $\Intro{\wedge}$ 4,6
    \end{logicproof}

    $$p \wedge (q \wedge r) \vdash (p \wedge q) \wedge r$$

    \begin{logicproof}{3}
        p \wedge (q \wedge r) & premise \\
        q \wedge r & $\ElimB{\wedge}$ 1 \\
        p & $\ElimA{\wedge}$ 1 \\ 
        q & $\ElimA{\wedge}$ 2 \\ 
        r & $\ElimB{\wedge}$ 2 \\ 
        p \wedge q & $\Intro{\wedge}$ 3,4 \\
        (p \wedge q) \wedge r & $\Intro{\wedge}$ 6,5
    \end{logicproof}

    $$p \wedge q \vdash q \wedge p$$

    \begin{logicproof}{3}
        p \wedge q & premise \\
        p & $\ElimA{\wedge}$ 1 \\
        q & $\ElimB{\wedge}$ 1 \\
        q \wedge p & $\Intro{\wedge}$ 3,2
    \end{logicproof}

    Now that we have proven $\wedge$ is associative and commutative, we can rewrite the statement as $\left( \bigwedge_{i=1}^{n} \varphi_{i} \right) \to \psi \vdash \varphi_{1} \to (\cdots \to (\varphi_{n} \to \psi))$.  
    
    \noindent We'll show it all here in one presentation for convenience.  For the base case, we wish to prove the case where there's only one $\varphi$.  That is, we want to prove $\varphi_{1} \to \psi \vdash \varphi_{1} \to \psi$, but the proof is silly as the statement to show is the premise itself.  

    \begin{logicproof}{3}
        \varphi_{1} \to \psi & premise 
    \end{logicproof}

    \iffalse 
    \noindent Moving on to the induction step.  Suppose $n \in \mathbb{Z}^+$ and assume that the theorem is true for $n - 1$ $\varphi$ variables and show that it's true for $n$ $\varphi$ variables.  That is, assume $(\varphi_{n-1} \wedge ( \varphi_{n-2} \wedge \cdots (\varphi_{2} \wedge \varphi_{1}) \cdots )) \to \psi \vdash \varphi_{n-1} \to (\varphi_{n-2} \to \cdots (\varphi_{2} \to (\varphi_{1} \to \psi)))$ holds.  Then we have 

    \begin{logicproof}{3}
        (\varphi_{1} \wedge ( \varphi_{2} \wedge \cdots (\varphi_{n-1} \wedge \varphi_{n}) \cdots )) \to \psi  & premise \\
        \begin{subproof}
            \varphi_{n} & assumption \\ 
            
        \end{subproof}
    \end{logicproof}
    \fi 
    \fi 
\end{proof}

\begin{problem}{1}
    Show that any of the three rules \{(LEM), (PBC), ($\neg \neg$E)\} are interderivable.  
\end{problem}

\begin{proof}
    First, we assume ($\neg \neg$E) and wish to show (PBC).  

    \begin{logicproof}{3}
        \neg \varphi \to \bot & premise \\
        \begin{subproof}
            \neg \varphi & assumption \\ 
            \bot & $\Elim{\to}$ 1,2 
        \end{subproof}
        \neg \neg \varphi & $\Intro{\neg}$ 2--3 \\
        \varphi & $\Elim{\neg \neg}$ 4
    \end{logicproof}

    We prove one of De Morgan's laws as a lemma before continuing.  Specifically we show:  

    $$\neg (p \vee q) \vdash \neg p \wedge \neg q$$

    \begin{logicproof}{3}
        \neg (p \vee q) & premise \\
        \begin{subproof}
            p & assumption \\ 
            p \vee q & $\IntroA{\vee}$ 2 \\
            \bot & $\Elim{\neg}$ 1,3 
        \end{subproof}
        \neg p & $\Intro{\neg}$ 2--4 \\
        \begin{subproof}
            q & assumption \\
            p \vee q & $\IntroB{\vee}$ 6 \\
            \bot & $\Elim{\neg}$ 1,7 
        \end{subproof}
        \neg q & $\Intro{\neg}$ 6--8 \\ 
        \neg p \wedge \neg q & $\Intro{\wedge}$ 5,9
    \end{logicproof}

    Next, we assume (PBC) and show (LEM).

    \begin{logicproof}{3}
        \begin{subproof}
            \neg (\varphi \vee \neg \varphi) & assumption \\
            \neg \varphi \wedge \neg \neg \varphi & De Morgan \\
            \neg \varphi & $\ElimA{\wedge}$ 2 \\ 
            \neg \neg \varphi & $\ElimB{\wedge}$ 2 \\ 
            \bot & $\Elim{\neg}$ 3,4
        \end{subproof}
        \varphi \vee \neg \varphi & PBC 
    \end{logicproof}

    Finally, we assume (LEM) and show ($\neg \neg$E).  

    \begin{logicproof}{3}
        \neg \neg \varphi & premise \\
        \varphi \vee \neg \varphi & LEM \\ 
        \begin{subproof}
            \varphi & assumption 
        \end{subproof}
        \begin{subproof}
            \neg \varphi & assumption \\
            \bot & $\Elim{\neg}$ 1,4 \\ 
            \varphi & $\Elim{\bot}$ 5
        \end{subproof}
        \varphi & $\Elim{\vee}$ 2,3,4--6
    \end{logicproof}
\end{proof}
\end{document}